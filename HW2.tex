\documentclass[a4paper,12pt]{article} 
\usepackage[T2A]{fontenc}   
\usepackage[utf8]{inputenc}   
\usepackage[english,russian]{babel} 
\usepackage{amsmath,amsfonts,amssymb,amsthm,mathrsfs,mathtools} 
\usepackage{cancel}
\usepackage{multirow}
\usepackage[colorlinks, linkcolor = blue]{hyperref}
\usepackage{upgreek}\usepackage[left=2cm,right=2cm,top=2cm,bottom=3cm,bindingoffset=0cm]{geometry}
\usepackage{tikz}
\usepackage{graphicx}
\usepackage{subfig}
\usepackage{titletoc}
\usepackage{pgfplots}
\usepackage{hhline}
\usepackage{xcolor}
\usepackage{wrapfig}
\newcommand{\angstrom}{\text{\normalfont\AA}}
\author{Бондарь София\\
Группа Б05-021}
\title{Домашнее задание №2}
\date{\vspace{-10pt}}


\begin{document}
\maketitle


\section*{Двойственность}
\section*{Задача №1}
\subsection*{Условие:}
Выведите двойственную задачу для задачи:
    \begin{equation*}
        \min_{\mathbf{x}\in\mathbb{R}^n} \|A\mathbf{x}-\mathbf{b}\|_2+\frac{\lambda}{2}\|\mathbf{x}\|_2^2,
    \end{equation*}
    введя новую переменную $\mathbf{y}\in\mathbb{R}^m$, такую что $\mathbf{y}=A\mathbf{x}-\mathbf{b}$ и соответствующие ограничения. Параметры: $A\in\mathbb{R}^{m\times n}$, $\lambda \geq 0$. Таким образом, мы получим задачу, которая дает нижнюю оценку на решение для задачи безусловной минимизации.
\subsection*{Решение:}
Рассмотрим функцию Лагранжа : $L(x, y, \nu) = ||y||_2 + \lambda/2 ||x||_2^2 + \nu^T(Ax-b-y)$ . 
Тогда при $x \in R^n$ инфинум данной функции будет равен: \\ 

$ inf L = - \nu^T + inf (||y||_2 + \lambda/2 ||x||_2^2 + \nu^T(Ax-y)) $ \\

$ L'_x = 0 <=> \lambda x^T + \nu^TA = 0 $ 

$ L'_y = 0 <=> y/||y|| = \nu $ 
\\ 
тогда при $x = -A^T\nu / \lambda$ достигается минимум функции и, если выполнено условие $  \nu \leq $ 1, то достигается минимум функции в нуле. Теперь рассмотрим снова выражение, в которое подставим полученные значения \\ 

$ inf L = - \nu^Tb + inf (||y||_2 - \nu^Ty) - \frac{1}{2\lambda} \nu^T (AA^T)\nu $ \\

$ inf L = -\infty $ при $ \nu \leq 1 $, иначе $inf L = 0  => max (- \nu^Tb - \frac{1}{2\lambda} \nu^T (AA^T)\nu )$ \qed


\newpage
\section*{Задача №2}
\subsection*{Условие:}

$[8]$ Рассмотрим некоторую динамическую систему с дискретным временем, которая в каждый момент характеризуется некоторым состоянием $\mathbf{x}_t \in \mathbb{R}^n, t=\overline{0,T}.$ Переход из одного состояния системы в другое осуществляется при помощи выбираемого линейного преобразования:
        $$\mathbf{x}_{t+1}=A_t \mathbf{x}_t.$$
В данной задаче мы будем считать, что матрицы $A$ выбираются из некоторого конечного множества $\mathcal{A}\in\left\{A^{(1)}\dots A^{(K)}\right\}\subset \mathbb{R}^{n \times n}.$ Наша задача выбрать последовательность матриц $\{A_t\}_{t=0}^{T-1}$, которая минимизирует суммарные "потери" за все время $t=\overline{0,T}$, т.е. мы хотим минимизировать функционал вида $\sum\limits_{i=1}^T f(\mathbf{x}_t)$ для некоторой заданной функции $f.$ Формально эту задачу можно записать в следующем виде:
        \begin{equation}
        \label{control_problem}
        \begin{aligned}
      \min_{\substack{\mathbf{x}_1\dots\mathbf{x}_T \in\mathbb{R}^n,\\ u_0\dots u_{T-1} \in\{1\dots K\}}}\sum\limits_{i=1}^T f(\mathbf{x}_t),\\
              \text{s.t.} \mathbf{x}_{t+1}=A^{(u_t)}\mathbf{x}_{t}, t=\overline{0, T-1}.
        \end{aligned}
    \end{equation} 
Начальное состояние $\mathbf{x}_0$ считаем данным. Функция $f$ для дальнейшего исследования не обязательно должна быть выпуклой, но мы будем считать, что нам известна ее сопряженная функция $f^*(\mathbf{y})=\sup_{\mathbf{x}\in\mathbb{R}^n}\left(\mathbf{y}^\top \mathbf{x}-f(\mathbf{x})\right).$

Данная задача является невыпуклой и достаточно тяжело решить её. Вместо этого, мы рассмотрим эвристический подход, основанный на двойственности.
\begin{enumerate}
    \item \label{a} $[2]$ Найдите двойственную задачу для задачи \eqref{control_problem}. Выражение для двойственной задачи может содержать $\mathbf{x}_0$,  $A^{(1)}\dots A^{(K)}$, функцию $f$ и её сопряженную $f^*.$
    \item \label{b} $[1]$ Пусть $\nu_1^*\dots \nu_T^* \in\mathbb{R}^T$ есть оптимальное значение двойственной переменной, соответствующей $T$ ограничениям исходной задачи. По данным $\nu_1^*\dots \nu_T^*$ мы будем искать $u_t$ следующим образом:
    $$(\tilde{u}_0\dots \tilde{u}_{T-1}) \in \argmin\limits_{\{u\}_{t=0}^{T-1} \in [K]^T } \inf\limits_{x_1\dots x_T} L\left(x_1\dots x_T, u_0\dots, x_{T-1}, \nu_1^* \dots,\nu_T^*\right).$$
    Опишите алгоритм нахождения $\tilde{u}_0\dots \tilde{u}_{T-1}$ таким способом. В данном пункте Вы должны описать, какие задачи оптимизации Вы решаете на каждом шаге и что Вы получаете в результате. Задачи оптимизации при построении алгоритма должны быть выписаны настолько просто, насколько возможно, и решены аналитически, если это возможно.
    \item  $[1]$ Рассмотрим случай $f(\mathbf{x})=\frac{1}{2}\mathbf{x}^\top Q \mathbf{x},$ где $Q\in\mathbb{S}^n_{++}$.  Найдите сопряженную функцию $f^*(\mathbf{y})$ для нее.
\end{enumerate}

\subsection*{Решение:}


\begin{enumerate}
    \item Рассмотрим функцию Лагранжа : \\
    $L(\{x_i\}_{i=1}^T, \{u_j\}_{j=0}^{T-1}, \{ \nu_k\}_{k=0}^{T-1} ) = \sum \limit_{i=1}^T f(x_i) + \sum \limit_{i=0}^{T-1} \nu_i^T(x_{i+1} - A^{u_i}x_i)$ \\

    Вытащим из первой суммы последнее слагаемое а из второй суммы первое:\\
    $L(\{x_i\}_{i=1}^T, \{u_j\}_{j=0}^{T-1}, \{ \nu_k\}_{k=0}^{T-1}) = \sum \limit_{i=1}^{T-1} (f(x_i) + (\nu_{i-1}^T - \nu_i^T A^{u_i})x_i) + f(x_T) - \nu_0^T A^{u_0}x_0 + \nu_{T-1}^T x_T.$ \\

    По условию $f^*(\mathbf{y})=\sup_{\mathbf{x}\in\mathbb{R}^n}\left(\mathbf{y}^\top \mathbf{x}-f(\mathbf{x})\right).$\\
    А это значит, что $-f^*(-\nu) = - \sup_{x}(-\nu^Tx - f(x)) = \inf_{x}(\nu^T x + f(x))$\\

    Возвращаясь к первому равенству, из второго тогда будет следовать:\\
    $\inf_{x}L(\{x_i\}_{i=1}^T, \{u_j\}_{j=0}^{T-1}, \{ \nu_k\}_{k=0}^{T-1} ) = - \nu_0^T A^{u_0}x_0 - f^*(\nu_{T-1}) - \sum \limits_{i=1}^{T-1} f^*(A^{u_i}\nu_i - \nu_{i-1})$

    Двойственная задача выражается так:\\
    $h(\nu) = - f^*(\nu_{T-1}) -\max_{A}(\nu_0^T Ax_0) - \sum \limits_{i=1}^{T-1} \max_{A}(f^*(A\nu_i - \nu_{i-1})) $ \qed

    \item Чтоб найти $\tilde{u}_t:$\\
    - Переберем все матрицы, так чтоб максимизировать следующие величины:\\
        - Для $t = 0$ хоттим максимизировать $\nu_0^T Ax_0$\\
        - Для остальных $t$ хотим максимизировать $f^*(A\nu_t - \nu_{t-1})$

    - Ответом является номер лучшей (для максимизации) матрицы

    \item  Если $f(\mathbf{x})=\frac{1}{2}\mathbf{x}^\top Q \mathbf{x}$, то \\
    $f^*(y) = \sup_{x} (y^Tx - f(x)) =>$\\

    $\nabla_x = y - Qx = 0 => x = Q^{-1}y =>$\\

    $f^*(y) = \frac{1}{2}y^TQ^{-1}y$ \qed
    
    
\end{enumerate}

\newpage
\section*{Условия оптимальности}
\section*{Задача №3}
\subsection*{Условие:}
Пусть параметры $\mathbf{a},\mathbf{b}\in\mathbb{R}^n_{++}$ имеют положительные компоненты,  при этом компоненты первого вектора отсортированы в порядке убывания $a_n\geq a_k \geq \dots a_1 > 0$, а компоненты второго вектора определены, как $b_k=\frac{1}{a_k}$. Выведите условия ККТ для задачи
        \begin{equation*}
        \begin{aligned}
      \min_{\mathbf{x}\in\mathbb{R}^n} -\log(\mathbf{a}^\top \mathbf{x})-\log(\mathbf{b}^\top \mathbf{x})\\
              \text{s.t. } \mathbf{x}\geq 0,\quad \mathb
              f{1}^\top \mathbf{x}=1
        \end{aligned}
    \end{equation*}  
    и покажите, что вектор $\mathbf{x}=\left(\frac{1}{2}, 0, 0\dots, 0, 0, \frac{1}{2}\right)^\top$ является решением этой задачи.
        \item $[1]$ Пусть $A\in\mathbb{S}^n_{++}$. Примените результат первой части задачи для вектора $\mathbf{a}=(\lambda_n\dots \lambda_1)^\top$, собственные значения в котором расположены в порядке убывания, чтобы доказать неравенство Канторовича:
        $$2\left(\mathbf{u}^\top A\mathbf{u}\right)^{\frac{1}{2}}\left(\mathbf{u}^\top A^{-1}\mathbf{u}\right)^{\frac{1}{2}}\leq \sqrt{\frac{\lambda_{n}}{\lambda_1}}+\sqrt{\frac{\lambda_{1}}{\lambda_n}},$$
        где вектор $\mathbf{u}\in\mathbb{R}^n, \|\mathbf{u}\|_2=1$.
        
        Hint: Если $A \mathbf{v}=\lambda \mathbf{v}$ и $A$ обратимая матрица, то $A^{-1}\mathbf{v} = \frac{1}{\lambda} \mathbf{v}$.

\subsection*{Решение:}

\\1. a. Задача  имеет вид \\
min $f_0(x) = -log(a^Tx) - log(b^Tx),$ s.t. $f_i(x) = -x_i \leq 0 $, $h(x) = \sum x_i - 1 = 0 $ \\

\\При этом это задача выпуклой оптимизации ($f_i(x) - $ выпуклые, h(x) - афинная -log ($a^Tx$) и - log($b^Tx$)  выпуклые), а значит ККТ - достаточные условия \\

ККТ:\\
$f_i(x )\leq 0 $\\
h(x) = 0\\
$\lambda_i^* \geq 0 $\\
$\lambda_i^* f_i(x) = 0$\\
$\nabla f_0(x^*) + \sum \lambda_i^* \nabla f_i(x^*) + \nu^* \nabla h(x^*) = 0 $\\ 

b. Преобразуя последнее уравнение получим ( e - единичный вектор, $e_i$ - вектор со всеми нулями и 1 на i месте) $- \frac{a}{a^Tx} - \frac{b}{b^Tx} - \sum \lambda_i^* + \nu^* = 0 $ \\

Тогда $x^* = (1/2, 0 ... 0, 1/2)^T$ удовлетворяет ККТ при $\lambda_2^* = 0, \lambda_i^*= - \frac{a}{a^Tx} - \frac{b}{b^Tx} + 2$ (i от 2 до n-1)$, \nu^* = 2$ \\

1 и 2 условия выполняются \\
условие 3: при i = 1 , n выполняется равенство, при других:\\

$\lambda_i^* = 2 - \frac{2a_i^2 + 2a_1a_n}{a_i(a_1 + a_n)} \geq 0$\\
$(a_n - a_1)(a_i - a_1) \geq 0$ - выполнено\\

4 условие: при i от 2 до n-1 $f_i(x^*) = 0$, те 4 тоже выполнено\\

Подстановкой $x^* $ в уравнение 5 ККТ убеждаемся, что i компонента обнуляется при i от 2 до n-1. При i = 1 , n $b_i=1/a_i$ \\

$- \frac{2a_i}{a_1+a_n} - \frac{2a_1a_2}{a_i(a_1+a_2)} + 2 = 0 $ равенство выполняется \\
тогда уравнение 5 выполняется. \\ 
чтд \\

2. $A=O^T\sum O$, $v = Ou$ => $u^TA^{-1} = v^T \sum ^{-1}v = \sum v_i^2/\lambda_i $ \\ love sonya 

тогда неравенство: $2(v^Ta)^{1/2}(v^Tb)^{1/2} \leq (\lambda_1 + \lambda_n)^{1/2}(1/\lambda_1 + 1/\lambda_n)^{1/2}$ \\

логарифмируем $ln(v^Ta) + \frac{1}{2}ln(v^Tb) \leq \frac{1}{2}(ln(\lambda_1 + \lambda_n) + ln(1/\lambda_1 + 1/\lambda_n) ) $ \\

$-ln(v^Ta) - \frac{1}{2}ln(v^Tb) \geq - ln(x^*^Ta) - ln(x^*^Tb)$ \\

тогда $x^*$ - минимум $f_0$ ( тк v, a, b удовлетворяет условиям пункта а) => $f_0(x^*) \leq f_0(v)$ \\ \qed







\newpage
\section*{Задача №4}
Рассмотрим задачу проекции на симплекс:
    \begin{equation*}
        \begin{aligned}
      \min_{\mathbf{x}\in\mathbb{R}^n} \|\mathbf{x}-\mathbf{y}\|^2 \\
        \text{s.t. }\sum\limits_{i=1}^n x_i=1, \mathbf{x}\geq 0.
        \end{aligned}
    \end{equation*}
 \begin{enumerate}
 \item $[2]$ Выведите условия ККТ, упростите их и сведите задачу к нахождению двойственной переменной $\nu\in\mathbb{R},$ соответствующей единственному ограничению равенству в исходной задаче.
 \item $[1]$ Постройте алгоритм для нахождения оптимального $\nu$ с учетом зависимости $x(\nu)$, полученной в предыдущем пункте. Оцените сложность данного подхода.
 \end{enumerate}

\subsection*{Решение:}
\\
1) Для начала запишем необходимые условия ККТ: \\
 a. $x, \lambda \geq 0$ \\
 b.  $\sum x_i$ = 1 \\
 c. $\lambda_ix_i = 0 $ \\
 d. $2x - 2y - \lambda + \nu e = 0 $ 
\\
из этих условий следует, что $x_i = max(0, y_i + \nu e/2)$. \\ Для таких х выполнены всу условия, кроме условия b. Для его выполнения необходимо выполнение следующего равенства: $\sum max(0, y_i + \nu e/2) = 1$\\


2) Теперь рассмотрим нужный алгоритм для поиска оптимального значения, для этого введем функцию $l(\nu ) = \sum max(0, y_i + \nu e/2) = 1$. \\
a. рассмотим отрезок от 0 до C ( так как при рассмотрении функции нетрудно заметить, что при $\
nu = 0$ она тоже равна 0, а при $\nu = \infty$ она равна бесконечности и при это она монотонна на этом участке.)\\
b. выберем мелкость разбиения (M - разбиение) и разобьем наш отрезок \\
c. теперь будем рассматривать значения в получившихся точках (концы отрезка нашего разбиения). С помощью бинпоиска мы сможем найти решение. \\
На каждом шаге мы считаем значение функции l за O(n), бинпоиск будет работать за log(|M|) - итого сложность данного подхода \textbf{O(nlog(|M|))} \qed

\newpage

\section*{Задача №5}
\subsection*{Условие:}

В данной задаче мы докажем следующий геометрический результат:

\textit{ Пусть $C$ есть многогранник в $\mathbb{R}^n$ вида $C=\{\mathbf{x}| -1\leq \mathbf{a}_i^\top \mathbf{x}\leq 1, i=\overline{1,p}\},$ такой что $\sum\limits_{i=1}^p \mathbf{a}_i \mathbf{a}_i^\top \succeq 0$. Тогда рассмотрим эллипсоид $\mathcal{E}=\{\mathbf{x}| \mathbf{x}^\top Q^{-1} \mathbf{x}\leq 1\}, Q\in\mathbb{S}^n_{++}$ максимального объема, такой что он вписан в $C$, т.е. $\mathcal{E}\subseteq C.$ Тогда $\sqrt{n} \mathcal{E}=\{\mathbf{x}| \mathbf{x}^\top Q^{-1} \mathbf{x}\leq n\}$ содержит $C$.}

\item \label{Q_cond} $[1]$ Покажите, что $\mathcal{E}\subseteq C$, тогда и только тогда, когда  $\mathbf{a}_i^\top Q \mathbf{a}_i \leq 1, i=\overline{1,p}.$

\textit{Hint:} Может быть полезным представить эллипсоид в виде $\mathcal{E}=\{Q^{1/2} \mathbf{y}\| \|\mathbf{y}\|_2\leq 1\}.$

\item $[2]$ Объем эллипсоида $\mathcal{E}$ пропорционален $\left(\det Q\right)^{1/2}.$ Тогда согласно пункту \ref{Q_cond}, мы можем определить матрицу $Q$ эллипсоида 
 $\mathcal{E}$ как решение следующей задачи:
     \begin{equation}
     \label{Q_problem}
        \begin{aligned}
      \min_{Q \in \mathbb{S}^n} -\log\det Q \\
        \text{s.t. } \mathbf{a}_i^\top Q \mathbf{a}_i \leq 1, i=\overline{1,p}.
        \end{aligned}
    \end{equation}
    Пусть $Q$ является оптимальным решением задачи \eqref{Q_problem}. Тогда используя условия ККТ для этой задачи, покажите, что
$$\mathbf{x}\in C \Rightarrow \mathbf{x}^\top Q^{-1} \mathbf{x}\leq n,$$
т.е. $C\in \sqrt{n} \mathcal{E}.$

\subsection*{Решение:}

1. $\mathcal{E}=\{Q^{1/2} \mathbf{y}\| \|\mathbf{y}\|_2\leq 1\} \in C$ <=> $||Q^{1/2}a_i||_2$ = $sup |a_i^TQ^{1/2}y| \leq 1 $ при $||y||_2 \leq 1$ для i от 1 до p\\ 

2. введем функцию $g(\lambda) = inf_{Q \succeq 0}$ $L(Q, \lambda) = inf_{Q \succeq 0}$ $(log det Q^{-1} + tr ((\sum \lambda_i a_i a_i^T)Q) - \sum \lambda_i)$ \\

заметим, что inf (log det$X^{-1} + tr(XY)$) = log det Y + n, если $Y \succ 0$, иначе $- \infty$\\

$-X^{-1}+Y = 0 => X=Y^{-1} $ при $Y \succ 0$, иначе существует ненулевое а, такое что $a^TYa\leq 0 $ при $X = I + taa^T$ получим $X = 1 + t||a||_2^2 $ и log det $X^{-1}$ + tr(XY) = - log($1+ta^Ta$) + trY + tr$a^TYa$ \\

тогда двойственная функция g($\lambda$) = log det $\sum (\lambda_i a_i a_i^T) - \sum \lampda_i + n$ при $\sum (\lambda_i a_i a_i^T) \succ 0$ , иначе $- \infty$\\

итого задача log det $\sum (\lambda_i a_i a_i^T) - \sum \lampda_i + n$ -> max, s.t. $\lambda \succeq 0$ \\

KKT: \\
$Q \succ 0$ \\
$a_i^TQa_i \leq 1 $ \\
$\lambda \succeq 0$ \\
$\lambda_i(1 - a_i^TQa_i ) = 0$ \\
$Q^{-1} = \sum \lambda_i a_i a_i^T$ \\

тогда $n = \sum \lambda_i tr(Q a_i a_i^T) = \sum \lambda_i$ => $x^TQ^{-1}x = \sum \lambda_i(a_i^T x)^2 \leq \sum \lambda_i = n $  \qed

\newpage
\section*{Conic Duality}
\section*{Задача №6}
\subsection*{Условие:}

Выведите двойственную задачу для задачи:
        \begin{equation*}
        \begin{aligned}
      \min_{\mathbf{x}\in\mathbb{R}^n} \|A\mathbf{x}-\mathbf{b}\|_2^2 \\
        \text{s.t. } \|\mathbf{x}\|_1\leq \alpha; \|\mathbf{x}\|_3\leq \beta; \|\mathbf{x}\|_{\infty}\leq \gamma,
        \end{aligned}
    \end{equation*}
где $A\in\mathbb{R}^{m\times n}$, $\alpha,\beta,\gamma>0$ - некоторые положительные константы.

\subsection*{Решение:} \\

Мы можем расписать условие через скалярные произведения двух векторов, например для $\|\mathbf{x}\|_1\leq \alpha$ - это будет входит в Лагранжиан как $\langle \begin{pmatrix} x \\ \alpha \end{pmatrix}; \begin{pmatrix} \lambda_1 \\ \nu_1 \end{pmatrix} \rangle$, где $\lambda_i - $ вектор, а $\nu_i - $ число.

Тогда $L(x, \lambda, \nu) = \|A\mathbf{x}-\mathbf{b}\|_2^2 - (\lambda_1^T + \lambda_2^T + \lambda_3^T) x - \nu_1 \alpha - \nu_2 \beta - \nu_3 \gamma => 0 = \nabla_x = 2A^T(Ax -b) - \lambda_i =>\frac{1}{2} \sum \limits_{i=1}^3 \lambda_i + A^Tb = A^TAx => x = (A^TA)^{-1}(\frac{1}{2} \sum \limits_{i=1}^3 \lambda_i + A^Tb)$

Двойственная задача получится подставлением найденного x в $L(x, \lambda, \nu)$ с ограниченииями на нормы $\lambda_i$ относительно $\nu_i$ \qed

\newpage
\section*{Задача №7}
\subsection*{Условие:}

 \label{maxcut} $[6]$ На семинарах мы рассмотрели задачу вида:
        \begin{equation}
        \label{maxcut_primal}
        \begin{aligned}
      \min_{\mathbf{x}\in \{-1, 1\}^n } \mathbf{x}^\top W \mathbf{x},
      \end{aligned}
    \end{equation}
    где $W\in\mathbb{S}^n$. Мы также вывели для нее двойственную задачу:
            \begin{equation}
        \label{maxcut_dual}
        \begin{aligned}
      \max_{\nu\in\mathbb{R}^n} -\mathbf{1}^\top \nu,\\
      \text{s.t. } W+\text{diag}(\nu)\succeq 0.
        \end{aligned}
    \end{equation}
Данная задача дает оценку снизу на оптимальное значение \eqref{maxcut_primal}. 
\begin{enumerate}
\item $[2]$ Покажите, что двойственная задача для задачи  \eqref{maxcut_dual} имеет вид:
            \begin{equation}
        \label{maxcut_dual_dual}
        \begin{aligned}
      \min_{X \in \mathbb{S}^n } \text{tr}(WX),\\
      \text{s.t. } X\succeq 0,\quad X_{ii}=1,i=\overline{1,n}.
        \end{aligned}
    \end{equation}
    Покажите, что если оптимальный $X$ в задаче \eqref{maxcut_dual_dual} имеет ранг 1, т.е. $\exists \mathbf{x}: X=\mathbf{x}\mathbf{x}^\top,$ то $\mathbf{x}$ - есть оптимальное решение задачи \eqref{maxcut_primal}
\item $[1]$ Как соотносятся оптимальные значения задач \eqref{maxcut_primal}, \eqref{maxcut_dual} и \eqref{maxcut_dual_dual}?
\item \label{cvxpy_dual_dual1} $[2]$ Решите задачу \eqref{maxcut_dual_dual} при помощи CvxPy. 
\item $[1]$ Восстановите приблизительное решение исходной задачи при помощи собственного разложения решения $X$ из пункта \ref{cvxpy_dual_dual1}. Пусть $\mathbf{v}$ есть собственный вектор матрицы $X$, соотвествующий максимальному собстенному значению. Тогда возьмем в качетсве аппроксимации решения исходной задачи $\hat{\mathbf{x}}=\text{sign} (\mathbf{v}).$ Сравните значение исходной задачи для такого вектора и полученное оптимальное значение задачи \eqref{maxcut_dual_dual}.


\end{enumerate}

\subsection*{Решение}

1. Введем функцию $L(\nu , x ) = 1^T\nu - tr(XW + Xdiag(\nu)) = \sum \nu_i(1-X_{ii}) -tr(XW) $, \\ таким образом мы получаем задачу $min( tr(WX))$ (тк она эквивалентна задаче $max (- tr(WX))$), $s.t. X \succeq 0$, $X_{ii}=1$, i от 1 до n \\
Заметим, что выполнено $(xx^T)_{ii} = x_i^2 $ и $tr(Wxx^T) = x^TWx$ , тогда если оптимальный $X$ в задаче \eqref{maxcut_dual_dual} имеет ранг 1, т.е. $\exists \mathbf{x}: X=\mathbf{x}\mathbf{x}^\top,$ то $\mathbf{x}$ - есть оптимальное решение задачи \eqref{maxcut_primal} \\
\\
2. Пусть Х будет оптимальным решением задачи (7), тогда Х (при ранге 1) и оптимален в задаче (5).  Нижняя оценка оптимального решения задачи (3) будет совпадать с оптимальными решениями задач (4), (5)


\end{document}